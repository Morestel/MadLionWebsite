\documentclass[a4paper, 12pt]{article}
\usepackage[utf8]{inputenc}
\usepackage[T1]{fontenc}
\usepackage[french]{babel}
\usepackage{graphicx}
\usepackage{amsmath}
\usepackage{amssymb}

\usepackage{hyperref}

\pagestyle{headings}

\title{Travaux pratiques MySQL 1}
\author{Thomas MORESTEL}
\date{\today}

\begin{document}

\maketitle

\begin{abstract}

Travaux pratique numéro 1

\end{abstract}

\newpage

Question 2:
On récupère le nom de notre table à l'aide de la commande 'SHOW DATABASE' puis 'USE /notre identifiant/' et enfin 'SHOW TABLES' pour voir si tut à fonctionner correctement.
\newline
Question 3:
On regarde à l'intérieur de notre table à l'aide de la commande 'DESCRIBE', on choisit où insérer notre clé primaire puis soit on effectue un 'ALTER TABLE' $<nom de la table>$ ADD PRIMARY KEY $<La ou on veut mettre la table>$ ou alors lors du processus de création de table, on écrit 'PRIMARY KEY' en face.
\newline
Pour les clés étrangères, en face du tuple on écrit 'REFERENCES' puis l'adresse où la clé fait référence.

Certaines valeurs sont connus d'avance comme le nom des médailles. Ainsi, on peut faire un 'ENUM('Element1', 'Element2', ...) 
\newline
Question 4:

Pour tester les contraites de clés étrangères, en début de document il faut écrire $'SET  FOREIGN\_KEY\_CHECKS=1;'$

J'avais inséré un nombre d'une nation n'existant pas pour un athlète, et lorsque j'ai sélectionné la table où il aurait dû appaître il n'y est pas car la valeur qui a été insérée n'est pas valide. De plus, de nombreuses erreurs s'étaient rajoutées concernant les clés étrangères.

En désactivant la vérification, toutes les erreurs ont disparus et l'athlète est réapparu dans la table.

Lorsque je réactive la contrainte, toutes les erreurs sont revenus. 
\newline
Question 5:

UPDATE resultat
SET medaille = 'O'
WHERE numathlete=(SELECT numathlete FROM athlete WHERE nom='David Douillet');

On effectue un 'SELECT * FROM resultat;' avant et après la modification pour voir si elle s'est bien déroulée.
\newline
Question 6:

a-
SELECT medaille
FROM resultat, sport
WHERE nom='Tennis'AND numjeux=(SELECT numjeux FROM jeux WHERE type='E' AND ville='Tokyo' AND annee=1964);
\newline
b-
SELECT medaille, jeux.ville, jeux.annee, athlete.nom
FROM resultat, jeux, athlete
WHERE resultat.numjeux=jeux.numjeux AND resultat.numathlete=athlete.numathlete AND athlete.nom='Carl Lewis';
\newline
c-
SELECT DISTINCT ville, annee
FROM jeux, athlete, resultat
WHERE resultat.numjeux=jeux.numjeux AND resultat.numathlete=athlete.numathlete AND athlete.nom LIKE "D%" OR athlete.nom LIKE "L%";
\newline
d-
SELECT ville, annee
FROM jeux, nation, participe, resultat, sport, epreuve
WHERE epreuve.numsport=sport.numsport AND epreuve.numepreuve=resultat.numepreuve AND jeux.numjeux=resultat.numjeux AND resultat.numjeux=participe.numjeux AND participe.numnation=nation.numnation AND  nation.nom='Ethiopie' AND resultat.medaille='O' AND sport.nom LIKE 'Athletisme'; 

e-
SELECT ville, annee
FROM nation, participe, jeux, resultat, epreuve
WHERE nation.numnation=participe.numnation AND jeux.numjeux=participe.numjeux AND jeux.numjeux=resultat.numjeux AND epreuve.numepreuve=resultat.numepreuve AND epreuve.typeclass LIKE 'E' AND nation.nom LIKE 'France' AND resultat.medaille LIKE 'O';

\newline
f-
SELECT COUNT(medaille), continent
FROM nation, resultat, participe, jeux
WHERE nation.numnation=participe.numnation AND jeux.numjeux=participe.numjeux AND resultat.numjeux=participe.numjeux
GROUP BY continent
ORDER BY COUNT(medaille) DESC;

\newline

SELECT continent, COUNT(medaille)
FROM nation, resultat, participe, jeux
WHERE nation.numnation=participe.numnation AND jeux.numjeux=participe.numjeux AND resultat.numjeux=participe.numjeux
GROUP BY continent
ORDER BY COUNT(medaille) DESC;


\end{document}
