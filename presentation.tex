\documentclass[12pt,a4paper]{article}
%le pr�ambule

\usepackage[T1]{fontenc}
\usepackage[francais]{babel}
\usepackage{url} % permet l'utilisation de la balise url pour les liens internet
\usepackage{graphicx} % permet l'utilisation des images

\title{Introduction � Linux} % titre de l'article
\author{auteurs} % auteur de l'article
\date{date} % date de l'article

%document principal
\begin{document}

\maketitle

\tableofcontents

\section{Le terminal}
\subsection{Qu'est-ce que le terminal ?}
  "Le terminal est un programme qui permet d'ouvrir une 'console' dans une
  interface graphique. Il permet de lancer des commandes."\\
  (\url{http://doc.ubuntu-fr.org/terminal})
  \begin{center}
   \includegraphics[width=350pt]{terminal.eps}
  \end{center}
\subsection{L'historique des commandes}
  Toutes les commandes qui sont tap�es dans le terminal sont enregistr�es. Il est possible de revenir aux commandes pr�c�dentes en utilisant les fl�ches de direction 'haut' et 'bas'.

\subsection{La compl�tion automatique des commandes}
  Pour �viter de taper une commande en entier, il est possible d'utiliser la compl�tion automatique. Pour cela, il suffit simplement d'appuyer sur la touche de tabulation situ�e � gauche de la lettre 'A'.

\section{Les commandes de base}
\subsection{Gestion de son arborescence}
  Afin de manipuler son arborescence, il faut conna�tre les commandes suivantes :
  \begin{itemize}
  \item pwd : pour savoir le chemin absolu du r�pertoire courant ;
  \item ls : pour lister le contenu du r�pertoire courant ;
  \item mkdir : pour cr�er un r�pertoire ;
  \item rm : pour supprimer des fichiers ;
  \item cd : pour changer de r�pertoire.
  \end{itemize}
 
\subsection{Manipuler les fichiers}
  Voici une liste de commandes utiles pour manipuler les fichiers :
  \begin{center}
   \begin{tabular}{|c|c|} %tableau de 2 colonnes, dont le contenu est centr� (c) et avec des bordures |
   \hline
   file & conna�tre le type du fichier ind�pendamment de son extension \\
   \hline
   cat, more, less & afficher le contenu d'un fichier \\
   \hline
   emacs & �diter des fichiers texte \\
   \hline
   \end{tabular}
  \end{center}
  
\section{Sp�cificit�s des fichiers \LaTeX}
 \subsection{Avantages}
  \LaTeX est un langage de programmation qui poss�de plusieurs avantages :
  \begin{itemize}
   \item il est possible d'ajouter des extensions gr�ce � des packages ;
   \item les formats de sortie sont divers : dvi, pdf, html ;
   \item permet de g�rer de tr�s gros fichiers texte ;
   \item la mise en page est automatique.
  \end{itemize}
 \subsection{Inconv�nients}
  \begin{enumerate}
   \item ce n'est pas une solution \textsc{wysiwyg} ;
   \item il est n�cessaire d'apprendre le nom des balises ;
   \item la mise en page est automatique.
  \end{enumerate}

 \subsection{Divers}
  Quelques points int�ressants :
  \begin{itemize}
   \item il est possible d'�crire des formules math�matiques format�es~: $$\sum_{i=1}^{n}{i}=\frac{n(n-1)}{2}$$
   \item il est possible de faire des pr�sentations~:\\ \url{http://latex-beamer.sourceforge.net} ;
   \item une introduction � \LaTeX :\\ \url{http://www.ctan.org/tex-archive/info/lshort/}.
  \end{itemize}

\end{document}
